%----------------------------------------------------------------------------------------
%	SECTION TITLE
%----------------------------------------------------------------------------------------

\cvsection{工作经历}

%----------------------------------------------------------------------------------------
%	SECTION CONTENT
%----------------------------------------------------------------------------------------

\begin{cventries}
	
\cventrycompany
{米哈游} % Organization
{上海} % Location
\cventryproject
{Project SH} % Job title
{10月~ 2021 - 11月~ 2022} % Date(s)
{PJSH是一款二次元开放世界TPS射击游戏。PJSH使用UE4引擎,业务逻辑代码使用C++。本人主要负责战斗系统的开发和维护。}
{ % Description(s) of tasks/responsibilities
	\begin{cvitems}
		% 技能节点类似守望先锋,包含 Entry,Condition,Action 和 State 节点
		% 问题,1. 数据流的问题 FVariant,进行序列化反序列化
		% 2. 同步问题,已节点为单位的同步
		% 3. 不支持循环
		% 4. 解决方式,使用 client authority,把问题占时掩盖了
		% 技能架构 sub node memory,支持节点重入
		% Skill Info Context, 队友技能P
		% 属性系统, add, set, override, 支持add multi,加成P
		% 技能同步是怎么实现的,有二个版本,第一个是客户端和服务器运行技能图,Simulate 短不会跑整个技能图,对于 Simulate 端,同步的粒度是以节点为颗粒度的。因为 simulate 端不跑技能的执行流,所以执行的时候,服务器还会提供节点执行的上下文信息。
		\item {\textbf{负责技能系统的开发和维护}。技能系统使用状态同步,通过技能图组织整个技能流程。本人主要负责开发和维护技能架构、技能同步、技能节点(超过180个)、Buff系统和属性系统。 }
		% jinja2
		\item{\textbf{节点自动生成工具}。使用Python脚本解析节点定义的proto文件,用Jinja2模板语言自动生成Runtime的节点C++代码,主要包括技能节点类和相关同步代码。其他程序只需要定义proto并实现接口即可添加节点。}
		% 旋转
		% 优化, FVarint 优化
		% 内存优化, 注册 Delegate,存放在技能组件里,大部分情况都是用不到,使用 Module,能够动态加载
		% 尽量改成事件驱动。技能在不在运行的时候,把 Tick 关掉。    
		\item{\textbf{Gameplay业务功能相关开发}。开发游戏核心特色玩法:化学反应系统、超能力时间缓速功能(TimeScale)等。}
	\end{cvitems}
}


\cventrycompany
{网易(杭州)网络有限公司~~~高级开发工程师} % Organization
{杭州,浙江} % Location

\cventrycompany
{绩效:A} % Organization
{~} % Location

\cventryproject
{F工作室H69} % Job title
{6月~ 2019 - 7月~ 2021} % Date(s)
{光明大陆2是一款北欧神话题材的西方魔幻MMORPG游戏,服务器使用BigWorld,客户端Neox,脚本Python。本人作为战斗负责人负责游戏战斗模块的开发和维护。}
{ % Description(s) of tasks/responsibilities
	\begin{cvitems}
		\item {\textbf{负责技能系统的开发和维护}。技能系统用类似蓝图编辑的方式制作技能节点图。主要开发了技能图架构、技能同步、符文系统,开发和维护技能Buff、状态互斥等。策划使用这套技能系统配置了超过\textbf{1500}个技能,能满足策划大部分技能配置需求。}
		% jinja2
		\item{\textbf{技能编辑器开发和维护}。基于Sunshine UI框架开发节点编辑器,嵌入游戏Runtime窗口,通过RPC协议实现编辑器和游戏通讯。编辑器支持支持特效预览、技能热同步、技能模板、静态数据检查、用Jinja2自动化生成节点文档。编辑器使用得到策划好评。}
		% 旋转
		%\item{技能性能优化。对技能性能进行性能优化,流量优化,手感优化。技能系统和未优化前相比,技能系统整体性能提升20\%,角色创建速度提升30\%,技能流量减少30\%。}
		% \item{符文系统开发。开发符文系统用于修改技能效果,监听事件触发符文效果,修改角色属性等。}
		\item{\textbf{属性系统开发}。作为MMO核心的数值系统,通过实现响应式编程的方式实现了一套数值公式系统,并引入推拉模式提高计算性能。系统可应用于角色,装备,成就,符文等各类和数值相关模块。}
	\end{cvitems}
}

%\cventryproject
%{UGC编辑器} % Job title
%{7月~ 2021 - 今} % Date(s)
%{女娲是一个元世界游戏编辑器,由游戏编辑器和游戏展示平台组成。期望实现一个低上手门槛,高创作上限的编辑器。}
%{ % Description(s) of tasks/responsibilities
%	\begin{cvitems}
%		\item {负责动画组件开发。}
%	\end{cvitems}
%}
%------------------------------------------------

%------------------------------------------------

\cventrycompany
{杭州无端科技股份有限公司~~~游戏开发工程师} % Organization
{杭州,浙江} % Location

\cventryproject
{生死狙击2项目组} % Job title
{3月~ 2018 - 5月 2019} % Date(s)
% 什么是ECS, 为什么采用ECS
{生死狙击2是一款FPS端游。游戏的客户端和服务器采用Unity3D游戏引擎及采用ECS框架的Entitas进行开发。}
{ % Description(s) of tasks/responsibilities
\begin{cvitems}
\item {负责角色动画系统的开发和维护。主要内容包括基于Unity3D动画状态机的类似的逻辑动画状态机开发和设计,逻辑动画状态机支持状态的过渡、融合、切换、打断。支持动画组件的状态同步、动画回放和状态回滚。}
% 跳跃碰到障碍物处理,跳跃空中按wad轻微添加移动,碰到障碍物处理,斜面检测,自定义算法,检测边缘和台阶。
\item{负责角色移动模块的开发。开发移动系统组件实现移动和跳跃。为了处理碰撞体的旋转使用自定义CCD碰撞算法处理角色趴下、潜水和游泳的移动处理。}
% 旋转
% \item{角色动画骨骼后处理模块。角色左倾和右倾、枪随手旋转而旋转、角色上半身稳定、瞄准对齐。}
\end{cvitems}
}
%------------------------------------------------

\cventryproject
{生死狙击1项目组} % Organization
{12月~ 2017 - 3月~ 2018} % Location
{生死狙击是一款FPS页游。游戏使用Alternativa3D引擎,Flash的As3语言开发。}
{ % Description(s) of tasks/responsibilities
	\begin{cvitems}
		% lightmap diffuse, lighmap, fc =  diffuse.rgb * lightmap.rgb* lightmap.aaa 
		\item {骨骼动画查看器。基于开源页游引擎Away3D的AwayBuilder场景编辑器二次开发,修改引擎从左手坐标系转换到右手坐标系导致的动画资源导入,Agal(Adobe Graphics Assembly Language)shader代码问题。支持页游场景解析导入(lightmap、遮挡剔除和雾特效)。}
		\item {粒子特效转换工具。编写Unity3D的插件,把AS3粒子特效文件转化为Unity3D的Prefab文件。主要处理粒子发射器坐标转换(Y->Z轴)、生命周期、UV、Size、SpriteSheet等效果匹配。}
	\end{cvitems}
}

%--------------------------

%\cventryproject
%{生死狙击1项目组} % Organization
%{8月~ 2017 - 3月~ 2018} % Location
%{骨骼动画查看器是一款基于开源页游引擎Away3D的AwayBuilder场景编辑器修改而成的动画查看器,支持导入查看DAE骨骼动画、武器和角色绑定、导入查看游戏场景,导入查看游戏粒子特效(基于粒子特效库Flare3D)。该工具极大提高了美术的开发效率。}
%{ % Description(s) of tasks/responsibilities
%	\begin{cvitems}
%		% 
%		% lightmap diffuse, lighmap, fc =  diffuse.rgb * lightmap.rgb* lightmap.aaa 
%		\item {主要工作包括将away3D引擎从默认的左手系改成右手系(含投影矩阵,DAE顶点数据,agal代码等),修复away3D骨骼动画bug,添加武器和角色绑定、功能场景解析导入(lightmap、遮挡剔除和雾特效)。}
%		% 性能优化,agal 代码缓存
%		\item {负责页游粒子特效库(基于Flare3D)的维护和开发。}
%	\end{cvitems}
%}
%%------------------------------------------------    
%\cventryproject
%{生死狙击1项目组} % Organization
%{12月~ 2017 - 1月~ 2018} % Location
%{粒子特效转换工具。支持大部分页游粒子特效转换成Unity3D的粒子特效。节省了美术的制作特效的成本。}
%{ % Description(s) of tasks/responsibilities
%	\begin{cvitems}
%		% 
%		% lightmap diffuse, lighmap, fc =  diffuse.rgb * lightmap.rgb* lightmap.aaa 
%		\item {负责转换工具的开发。主要内容包括页游粒子特效的解析,粒子坐标转换(Flare3D库粒子发射器方向为Y轴,Unity3D粒子发射器方向为Z轴),粒子生命周期匹配,粒子发射器和发射方向匹配,粒子特效匹配(UV特效,Size特效,SpriteSheet特效等)。}
%		% 性能优化,agal 代码缓存
%	\end{cvitems}
%}


%------------------------------------------------

\end{cventries}

%------------------------------------------------
