%----------------------------------------------------------------------------------------
%	SECTION TITLE
%----------------------------------------------------------------------------------------

\cvsection{工作经历}

%----------------------------------------------------------------------------------------
%	SECTION CONTENT
%----------------------------------------------------------------------------------------

\begin{cventries}

\cventrycompany
{网易(杭州)网络有限公司~~~高级开发工程师} % Organization
{杭州,浙江} % Location

\cventryproject
{F工作室H69} % Job title
{6月~ 2019 - 7月~ 2021} % Date(s)
{光明大陆2是一款北欧神话题材的西方魔幻MMO游戏,游戏服务器使用BigWorld,客户端使用Neox,脚本层使用Python编写。本人作为战斗负责人负责游戏战斗模块的开发和维护。}
{ % Description(s) of tasks/responsibilities
	\begin{cvitems}
		\item {\textbf{负责技能系统的开发和维护}。技能系统用类似蓝图编辑的方式制作技能节点图,策划根据程序提供的各种功能节点,可以自由配置节点和节点之间的触发关系。主要开发了开发技能节点,技能客户端服务器同步、技能状态、状态互斥、符文系统等。策划使用这套技能系统配置了超过\textbf{1500}个技能,能满足策划大部分技能配置需求。}
		% jinja2
		\item{\textbf{技能编辑器开发和维护}。技能编辑器主要用于策划配置技能。除了基础的技能配置和编辑功能,为了提高开发效率开发了有特效预览、技能热同步、技能模板、静态数据检查、自动化生成节点文档等。相比游戏中的其他编辑器,用户体验更好。}
		% 旋转
		\item{技能性能优化。对技能性能进行性能优化,流量优化,手感优化。技能系统和未优化前相比,技能系统整体性能提升20\%,角色创建速度提升30\%,技能流量减少30\%。}
		% \item{符文系统开发。开发符文系统用于修改技能效果,监听事件触发符文效果,修改角色属性等。}
		\item{属性系统开发。属性系统主要用于服务角色的战斗数值系统,提供一个方便的数值存储和更新机制,减少各个模块之间的耦合。该系统还应用于装备,成就等各类养成模块,非常方便数值的拓展。}
	\end{cvitems}
}

\cventryproject
{UGC编辑器} % Job title
{7月~ 2021 - 今} % Date(s)
{女娲是一个元世界游戏编辑器,由游戏编辑器和游戏展示平台组成。期望实现一个低上手门槛,高创作上限的编辑器。}
{ % Description(s) of tasks/responsibilities
	\begin{cvitems}
		\item {负责动画组件开发。}
	\end{cvitems}
}
%------------------------------------------------

%------------------------------------------------

\cventrycompany
{杭州无端科技股份有限公司~~~游戏开发工程师} % Organization
{杭州,浙江} % Location

\cventryproject
{生死狙击2项目组} % Job title
{3月~ 2018 - 5月 2019} % Date(s)
% 什么是ECS, 为什么采用ECS
{生死狙击2是一款FPS端游,对标PUBG。游戏的客户端和服务器采用Unity3D游戏引擎及采用ECS框架的Entitas进行开发。}
{ % Description(s) of tasks/responsibilities
\begin{cvitems}
\item {负责角色动画系统的开发和维护。主要内容包括基于Unity3D动画状态机的游戏内部逻辑动画状态机开发和设计,内部状态机支持动画的过渡、融合、切换、打断、支持动画的同步和回放。动画的表现效果基本和吃鸡保持一致。}
% 跳跃碰到障碍物处理,跳跃空中按wad轻微添加移动,碰到障碍物处理,斜面检测,自定义算法,检测边缘和台阶。
\item{负责角色移动模块的开发。移动包括跳跃、站立、趴下、游泳和潜水移动。趴下、潜水和游泳移动使用了自定义的碰撞胶囊体,默认的碰撞胶囊体是朝向y轴,不适合趴下、潜水和游泳情况,实现了自定义的碰撞检测和移动算法,支持碰撞胶囊体朝向x,z轴,能够符合预期移动和旋转。}
% 旋转
\item{角色动画骨骼后处理模块。角色左倾和右倾、枪随手旋转而旋转、角色上半身稳定、瞄准对齐。}
\end{cvitems}
}
%------------------------------------------------

\cventryproject
{生死狙击1项目组} % Organization
{8月~ 2017 - 3月~ 2018} % Location
{骨骼动画查看器是一款基于开源页游引擎Away3D的AwayBuilder场景编辑器修改而成的动画查看器,支持导入查看DAE骨骼动画、武器和角色绑定、导入查看游戏场景,导入查看游戏粒子特效(基于粒子特效库Flare3D)。该工具极大提高了美术的开发效率。}
{ % Description(s) of tasks/responsibilities
	\begin{cvitems}
		% 
		% lightmap diffuse, lighmap, fc =  diffuse.rgb * lightmap.rgb* lightmap.aaa 
		\item {主要工作包括将away3D引擎从默认的左手系改成右手系(含投影矩阵,DAE顶点数据,agal代码等),修复away3D骨骼动画bug,添加武器和角色绑定、功能场景解析导入(lightmap、PVS、遮挡剔除和雾特效)。}
		% 性能优化,agal 代码缓存
		\item {负责页游粒子特效库(基于Flare3D)的维护和开发。}
	\end{cvitems}
}
%------------------------------------------------    
\cventryproject
{生死狙击1项目组} % Organization
{12月~ 2017 - 1月~ 2018} % Location
{粒子特效转换工具。支持大部分页游粒子特效转换成Unity3D的粒子特效。节省了美术的制作特效的成本。}
{ % Description(s) of tasks/responsibilities
	\begin{cvitems}
		% 
		% lightmap diffuse, lighmap, fc =  diffuse.rgb * lightmap.rgb* lightmap.aaa 
		\item {负责转换工具的开发。主要内容包括页游粒子特效的解析,粒子坐标转换(Flare3D库粒子发射器方向为Y轴,Unity3D粒子发射器方向为Z轴),粒子生命周期匹配,粒子发射器和发射方向匹配,粒子特效匹配(UV特效,Size特效,SpriteSheet特效等)。}
		% 性能优化,agal 代码缓存
	\end{cvitems}
}


%------------------------------------------------

\end{cventries}

%------------------------------------------------
