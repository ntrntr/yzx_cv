%----------------------------------------------------------------------------------------
%	SECTION TITLE
%----------------------------------------------------------------------------------------

\cvsection{工作经历}

%----------------------------------------------------------------------------------------
%	SECTION CONTENT
%----------------------------------------------------------------------------------------

\begin{cventries}

\cventrycompany
{网易(杭州)网络有限公司~~~高级开发工程师} % Organization
{杭州,浙江} % Location

\cventryproject
{F工作室H69} % Job title
{6月~ 2019 - 今} % Date(s)
{光明大陆2是一款MMOARPG游戏,游戏服务器使用BigWorld,客户端使用Neox,业务逻辑使用Python编写。本人作为战斗负责人负责游戏战斗模块的开发和维护。}
{ % Description(s) of tasks/responsibilities
	\begin{cvitems}
		\item {负责技能系统的开发和维护。技能系统用类似蓝图编辑的方式制作技能节点图,策划根据程序提供的各种功能节点,可以自由配置节点和节点之间的触发关系。主要工作是开发技能节点,主要有动画节点、法术场节点、状态节点、特效节点等来满足技能制作需求。此外实现技能客户端服务器同步、技能状态、状态互斥、双摇杆机制、法术场结算等。}
		% jinja2
		\item{技能编辑器开发和维护。实现编辑器的需求,提高编辑器的开发效率,主要开发功能有特效预览、技能热同步、技能模板、静态数据检查、自动化生成节点文档和编辑器META等。}
		% 旋转
		\item{技能性能优化。对技能性能进行性能优化,流量优化,手感优化。技能系统和未优化前相比,技能系统整体性能提升20\%,角色创建速度提升30\%,技能流量减少30\%。}
		\item{符文系统开发。开发符文系统用于修改技能效果,监听事件触发符文效果,修改角色属性等。}
		\item{属性系统开发。属性系统主要用于服务角色的战斗数值系统,提供一个方便的数值存储和更新机制。装备,成就,天赋都使用属性系统统一管理。}
	\end{cvitems}
}
%------------------------------------------------

%------------------------------------------------

\cventrycompany
{杭州无端科技股份有限公司~~~游戏开发工程师} % Organization
{杭州,浙江} % Location

\cventryproject
{生死狙击2项目组} % Job title
{3月~ 2018 - 5月 2019} % Date(s)
% 什么是ECS, 为什么采用ECS
{生死狙击2是一款FPS端游,包括了吃鸡模式和传统模式。游戏的客户端和服务器采用Unity3D游戏引擎及采用ECS框架的Entitas进行开发。}
{ % Description(s) of tasks/responsibilities
\begin{cvitems}
\item {负责角色动画系统的开发和维护。主要内容包括基于Unity3D动画状态机的游戏内部逻辑动画状态机开发和设计,内部状态机支持动画的过渡、融合、切换、打断、动画事件回调、动画网络同步和回放。熟悉使用Unity3D的IK和Avatar系统。}
% 跳跃碰到障碍物处理,跳跃空中按wad轻微添加移动,碰到障碍物处理,斜面检测,自定义算法,检测边缘和台阶。
\item{负责角色移动模块的开发。包括跳跃移动、角色站在地面超过一定角度下滑、站立移动、趴下移动、游泳移动和潜水移动。}
% 旋转
\item{角色动画骨骼后处理模块。角色左倾和右倾、枪随手旋转而旋转、角色上半身稳定、瞄准对齐。}
\item{负责角色自定义碰撞胶囊体开发。游戏默认的碰撞胶囊体是朝向y轴,采用自定义的碰撞检测和移动算法,支持碰撞胶囊体朝向x,z轴,能够正确沿着x,y轴旋转并且碰撞胶囊体碰到障碍物自动停止旋转,符合预期移动。自定义碰撞胶囊体应用于趴下、潜水和游泳移动。}
\end{cvitems}
}
%------------------------------------------------

\cventryproject
{生死狙击1项目组} % Organization
{8月~ 2017 - 3月~ 2018} % Location
{骨骼动画查看器是一款基于开源页游引擎Away3D的AwayBuilder场景编辑器修改而成的动画查看器,支持导入查看DAE骨骼动画、武器和角色绑定、导入查看游戏场景,导入查看游戏粒子特效(基于粒子特效库Flare3D)。该工具极大提高了美术的开发效率。}
{ % Description(s) of tasks/responsibilities
	\begin{cvitems}
		% 
		% lightmap diffuse, lighmap, fc =  diffuse.rgb * lightmap.rgb* lightmap.aaa 
		\item {主要工作包括将away3D引擎从默认的左手系改成右手系(含投影矩阵,DAE顶点数据,agal代码等),修复away3D骨骼动画bug,添加武器和角色绑定、功能场景解析导入、lightmap、PVS、遮挡剔除和雾特效。}
		% 性能优化,agal 代码缓存
		\item {负责页游粒子特效库(基于Flare3D)的维护和开发,包括粒子特效新特性,特效库性能优化等}
	\end{cvitems}
}
%------------------------------------------------    
\cventryproject
{生死狙击1项目组} % Organization
{12月~ 2017 - 1月~ 2018} % Location
{粒子特效转换工具。支持大部分页游粒子特效转换成Unity3D的粒子特效。节省了美术的制作成本。}
{ % Description(s) of tasks/responsibilities
	\begin{cvitems}
		% 
		% lightmap diffuse, lighmap, fc =  diffuse.rgb * lightmap.rgb* lightmap.aaa 
		\item {负责转换工具的开发。主要内容包括页游粒子特效的解析,粒子坐标转换(Flare3D库粒子发射器方向为Y轴,Unity3D粒子发射器方向为Z轴),粒子生命周期匹配,粒子发射器和发射方向匹配,粒子特效匹配(UV特效,Size特效,SpriteSheet特效等)。}
		% 性能优化,agal 代码缓存
	\end{cvitems}
}
%------------------------------------------------ 
\cventry
{软件开发工程师~~~实习} % Job title
{上海立时飞讯有限公司} % Organization
{上海} % Location
{6月 2016 - 9月 2016} % Date(s)
{ % Description(s) of tasks/responsibilities
\begin{cvitems}
\item {参与LFFTS(全文本搜索引擎)开发。原有通知消息队列方法只支持MSMQ,添加了新的消息队列RabbitMQ和亚马逊的SQS,并提供安全的账号密码管理策略。}
\item {对Tag数据查询添加hierarchical功能并且使用NUnit框架编写自动化测试代码。}
\item {对查询添加记录功能。记录查询参数和查询时间和查询结果保存到文件中。}
\item {为程序提供异常崩溃自动保存dump的功能。}
\end{cvitems}
}

%------------------------------------------------

\end{cventries}

%------------------------------------------------
